%% abtex2-modelo-artigo.tex, v-1.9.7 laurocesar
%% Copyright 2012-2018 by abnTeX2 group at http://www.abntex.net.br/ 
%%
%% This work may be distributed and/or modified under the
%% conditions of the LaTeX Project Public License, either version 1.3
%% of this license or (at your option) any later version.
%% The latest version of this license is in
%%   http://www.latex-project.org/lppl.txt
%% and version 1.3 or later is part of all distributions of LaTeX
%% version 2005/12/01 or later.
%%
%% This work has the LPPL maintenance status `maintained'.
%% 
%% The Current Maintainer of this work is the abnTeX2 team, led
%% by Lauro César Araujo. Further information are available on 
%% http://www.abntex.net.br/
%%
%% This work consists of the files abntex2-modelo-artigo.tex and
%% abntex2-modelo-references.bib
%%

% ------------------------------------------------------------------------
% ------------------------------------------------------------------------
% abnTeX2: Modelo de Artigo Acadêmico em conformidade com
% ABNT NBR 6022:2018: Informação e documentação - Artigo em publicação 
% periódica científica - Apresentação
% ------------------------------------------------------------------------
% ------------------------------------------------------------------------

\documentclass[
	% -- opções da classe memoir --
	article,			% indica que é um artigo acadêmico
	11pt,				% tamanho da fonte
	oneside,			% para impressão apenas no recto. Oposto a twoside
	a4paper,			% tamanho do papel. 
	% -- opções da classe abntex2 --
	%chapter=TITLE,		% títulos de capítulos convertidos em letras maiúsculas
	%section=TITLE,		% títulos de seções convertidos em letras maiúsculas
	%subsection=TITLE,	% títulos de subseções convertidos em letras maiúsculas
	%subsubsection=TITLE % títulos de subsubseções convertidos em letras maiúsculas
	% -- opções do pacote babel --
	english,			% idioma adicional para hifenização
	brazil,				% o último idioma é o principal do documento
	sumario=tradicional
]{abntex2}


% ---
% PACOTES
% ---

% ---
% Pacotes fundamentais 
% ---
\usepackage{lmodern}			   % Usa a fonte Latin Modern
\usepackage[T1]{fontenc}		% Seleção de códigos de fonte.
\usepackage[utf8]{inputenc}	% Codificação do documento (conversão
                              % automática dos acentos)
\usepackage{indentfirst}		% Indenta o primeiro parágrafo de cada seção.
\usepackage{nomencl} 			% Lista de símbolos
\usepackage{color}				% Controle das cores
\usepackage{graphicx}			% Inclusão de gráficos
\usepackage{microtype} 			% Para melhorias de justificação
% ---
		
% ---
% Pacotes de citações
% ---
\usepackage[brazilian,hyperpageref]{backref}	 % Paginas com as citações na bibl
\usepackage[alf]{abntex2cite}	% Citações padrão ABNT
% ---

% ---
% Configurações do pacote backref
% Usado sem a opção hyperpageref de backref
\renewcommand{\backrefpagesname}{Citado na(s) página(s):~}
% Texto padrão antes do número das páginas
\renewcommand{\backref}{}
% Define os textos da citação
\renewcommand*{\backrefalt}[4]{
	\ifcase #1 %
		Nenhuma citação no texto.%
	\or
		Citado na página #2.%
	\else
		Citado #1 vezes nas páginas #2.%
	\fi}%
% ---

% --- Informações de dados para CAPA e FOLHA DE ROSTO ---
\titulo{Educação em Direitos Humanos}
\tituloestrangeiro{Human Rights Education}
\autor{
   Fabrício Barros Cabral\\
   \tiny{Universidade Federal da Paraíba (UFPB)}
}
\local{Brasil}
\data{2023}
% ---

% ---
% Configurações de aparência do PDF final

% alterando o aspecto da cor azul
\definecolor{blue}{RGB}{41,5,195}

% informações do PDF
\makeatletter
\hypersetup{
     	%pagebackref=true,
		pdftitle={\@title}, 
		pdfauthor={\@author},
    	pdfsubject={Modelo de artigo científico com abnTeX2},
	    pdfcreator={LaTeX with abnTeX2},
		pdfkeywords={abnt}{latex}{abntex}{abntex2}{atigo científico}, 
		colorlinks=true,       		% false: boxed links; true: colored links
    	linkcolor=blue,          	% color of internal links
    	citecolor=blue,        		% color of links to bibliography
    	filecolor=magenta,      		% color of file links
		urlcolor=blue,
		bookmarksdepth=4
}
\makeatother
% --- 

% ---
% compila o indice
% ---
\makeindex
% ---

% ---
% Altera as margens padrões
% ---
\setlrmarginsandblock{3cm}{3cm}{*}
\setulmarginsandblock{3cm}{3cm}{*}
\checkandfixthelayout
% ---

% --- 
% Espaçamentos entre linhas e parágrafos 
% --- 

% O tamanho do parágrafo é dado por:
\setlength{\parindent}{1.3cm}

% Controle do espaçamento entre um parágrafo e outro:
\setlength{\parskip}{0.2cm}  % tente também \onelineskip

% Espaçamento simples
\SingleSpacing


% ----
% Início do documento
% ----
\begin{document}

% Seleciona o idioma do documento (conforme pacotes do babel)
%\selectlanguage{english}
\selectlanguage{brazil}

% Retira espaço extra obsoleto entre as frases.
\frenchspacing

% ----------------------------------------------------------
% ELEMENTOS PRÉ-TEXTUAIS
% ----------------------------------------------------------

%---
%
% Se desejar escrever o artigo em duas colunas, descomente a linha abaixo
% e a linha com o texto ``FIM DE ARTIGO EM DUAS COLUNAS''.
% \twocolumn[    		% INICIO DE ARTIGO EM DUAS COLUNAS
%
%---

% página de titulo principal (obrigatório)
\maketitle


% titulo em outro idioma (opcional)

% resumo em português
\begin{resumoumacoluna}
 Aqui vai um resumo em português do artigo.

 \vspace{\onelineskip}
 
 \noindent
 \textbf{Palavras-chave}: Direitos Humanos. Educação em Direitos Humanos.
\end{resumoumacoluna}


% resumo em inglês
\renewcommand{\resumoname}{Abstract}
\begin{resumoumacoluna}
 \begin{otherlanguage*}{english}
   Here goes an abstract of this paper.

   \vspace{\onelineskip}
 
   \noindent
   \textbf{Keywords}: Human Rights. Education on Human Rights.
 \end{otherlanguage*}  
\end{resumoumacoluna}

% ]  				% FIM DE ARTIGO EM DUAS COLUNAS
% ---

% ----------------------------------------------------------
% ELEMENTOS TEXTUAIS
% ----------------------------------------------------------
\textual

% ----------------------------------------------------------
% Introdução
% ----------------------------------------------------------
\section{Introdução}

As Nações Unidas reconheceram que os principais obstáculos para a realização do
direito ao desevolvimento e a paz são: todas as formas de racismo e
discriminação racial, a dominação e a ocupação estrangeira de territórios, a
agressão e as ameaças contra a soberania nacional, o repúdio ao reconhecimento
do direito ddos povos à autodeterminação e ao desenvolvimento sem intervenção
exterior, assim como todas as formas de escravidão, a corrida armamentista,
a ruína do meio ambiente, a dívida externa, a extrema pobreza, entre muitos
outros \cite{EducacaoDireitosHumanos}.

\noindent
O direito ao desenvolvimento não é apenas um direito fundamental, mas também uma
necessidade do ser humano, que corresponde às aspirações dos indivíduos e dos
povos de garantir em maior grau a liberdade e a dignidade
\cite{EducacaoDireitosHumanos}.

\noindent
O respeito aos direitos humanos é fundamental para o progresso e o
desenvolvimento social e econômico. Os instrumentos de proteção dos direitos
humanos tratam essencialmente da salvaguada de cada pessoa, grupos ou,
inclusive, nações \cite{EducacaoDireitosHumanos}.

\noindent
Entre os direitos contidos na \textit{Declaração Universal dos Direitos Humanos}
estão: o direito à vida, à liberdade e à segurança da pessoa; a igualdade
perante a lei; a liberdade de circular livremente e escolher sua residência; o
direito de não ser submetido a torturas nem penas ou tratamentos cruéis; o
direito de exercer o voto para escolha dos seus representantes e participar do
governo; o direito à educação, à assistência médica e ao trabalho; o direito à
propriedade; a liberdade de pensamento, consciência e religião; o direito à
previdência social, e o direito a um nível de vida adequado
\cite{DeclaracaoDireitosHumanos}.


% ----------------------------------------------------------
% Seção de explicações
% ----------------------------------------------------------
\section{Educar em Direitos Humanos e Formação de Educadores: Principais
Desafios}

\citeonline{EducacaoDireitosHumanosFormacaoEducadores} elenca que os principais
desafios para a formação de educadores em direitos humanos no contexto da 
``Educação em Direitos Humanos na América Latina e no Brasil: gênese histórica
e realidade atual'' são:

\begin{enumerate}
	\item Descontruir a visão do senso comum sobre os Direitos Humanos;
	\item Assumir uma concepção de educação em Direitos Humanos e explicitar
		  o que se pretende atingir em casa situação concreta;
	\item Articular ações de sesibilização e formação;
	\item Construir ambientes educativos que respeitem e promovam os Direitos
		  Humanos;
	\item Incorporar a educação em Direitos Humanos no currículo escolar;
	\item Introduzir a educação em Direitos Humanos na formação inicial e
		  continuada de educadores;
	\item Estimular a produção de materiais de apoio.
\end{enumerate}


% ---
% Finaliza a parte no bookmark do PDF, para que se inicie o bookmark na raiz
% ---
\bookmarksetup{startatroot}
% ---

% ---
% Conclusão
% ---
\section{Considerações finais}

\begin{citacao}
\end{citacao}

% ----------------------------------------------------------
% ELEMENTOS PÓS-TEXTUAIS
% ----------------------------------------------------------
\postextual

% ----------------------------------------------------------
% Referências bibliográficas
% ----------------------------------------------------------
\bibliography{referencias.bib}


% ----------------------------------------------------------
% Agradecimentos
% ----------------------------------------------------------
\section*{Agradecimentos}
O autor gostaria de agradecer a Profa. Dra. Ana Hermínia e a UFPB pelo apoio
para a confecção desse artigo.

\end{document}
