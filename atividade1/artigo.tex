\documentclass[12pt]{article}   % artigo fonte 12
\usepackage[
    a4paper,% papel A4
    left=3cm,% margem esquerda
    right=2cm,% margem direita
    top=3cm,% margem superior
    bottom=2cm% margem inferior
]{geometry}
% Caracteres / Acentos / Português do Brasil
\usepackage[utf8]{inputenc}
\usepackage[T1]{fontenc}
\usepackage[brazil]{babel}
\usepackage{ragged2e} %Usar: \justify
% Pacotes Matemática
\usepackage{array,latexsym}
\usepackage{amsmath,amsfonts,amssymb,amsthm,mathabx,amstext}
\usepackage{dsfont}  % Conjuntos: $\mathds{N, Z, Q, R, C}$

\begin{document}

\tableofcontents

\section{Variável Aleatória Contínua}

Uma variável $X$ é denominada de \textbf{variável aleatória contínua (v.a.c.)}
quando seu espaço amostral $R_X$ é um conjunto infinito não enumerável. Como
exemplos de variáveis aleatórias contínuas podemos citar:
\begin{itemize}
    \item resistência de um material,
    \item concentração de CO$_2$ na água,
    \item tempo de vida de um componente eletrônico,
    \item tempo de resposta de um sistema computacional,
    \item temperatura e medições (peso, altura, comprimento,...).
\end{itemize}

\subsection{Função de densidade}

Seja $X$ uma variável aleatória contínua (v.a.c). A função $f(x)$ que associa
a cada $x \in R_X$ um número real que satisfaz as seguintes condições:

\begin{enumerate}
    \item $f(x) \geqslant 0$, para todo $x \in R_X$ e
    \item $\displaystyle\int_{-\infty}^{+\infty}f(x) dx = 1,$
\end{enumerate}

\noindent é denominada de \textbf{função densidade de probabilidade (fdp)} da variável
aleatória $X$.
\par Neste caso $f(x)$ representa apenas a densidade no ponto $x$, ao
contrário da função de probabilidade $p(x)$ de uma variável aleatória discreta,
$f(x)$ aqui \textbf{não é a probabilidade} da variável $X$ assumir o valor $x$. Veremos a
seguir como se calcula probabilidades quando se tem uma distribuição contínua.

\subsection{Cálculo das Probabilidades}

Seja $X$ uma v.a.c com função densidade de probabilidade $f(x)$ Sejam $a < b$,
dois números reais. Define-se:

$$P(a < X < b) = \int_{a}^{b}f(x) dx,$$

\noindent isto é, a probabilidade de que $X$ assuma valores entre os números ``a'' e ``b''
é a área sob o gráfico da função $f(x)$ entre os pontos $x = a$ e $x = b$.

\end{document}

